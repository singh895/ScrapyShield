% Provide some information about the literature that you used to develop your project. Aim for a handful of paragraphs. 
% Don't go too hard. 
% If this is taking more than half a page (1-column), you're doing too much.

% This is a building project, so making sure that you provide a relationship between the work that you're doing and the topics from the class that's the most important part of this section.
% I'm adding one citation of an academic article here~\cite{dong2023behind} and a website here~\cite{least-weird-forum-user} so you know how citations look like.

% \subsection{Project Goals}

% Please outline what exactly are you building.
% This is probably easy if for example you have a good description of the GitHub issue or so.
% Please be as explicit as possible, don't aim to say too much.
% For example, if the task was ``we basically had to figure out where to put this one line of code but hoo boy was it hard'' that's great --- tell me that.

% background.tex
% \section{Background and Related Work}
Web crawling and security intersect in only a few niche studies. Classic XSS detection frameworks like GAXSS use genetic algorithms to evolve attack strings~\cite{dong2023behind}. Server‐side defenses against SQLi frequently employ parameterized queries and machine‐learning classifiers~\cite{least-weird-forum-user}, but these approaches focus on protecting applications, not the crawlers that consume them. Malware propagation via automated agents has been analyzed in “The Ghost in the Browser”~\cite{gupta2021ghost}, revealing how drive-by downloads can bypass naïve HTTP clients.

\subsection{Project Goals}
We aim to build a reusable, sandboxed testing harness that:
\begin{itemize}
  \item Simulates SQLi, XSS, and malware downloads in a controlled Flask environment.
  \item Adapts Scrapy spiders (including headless‐Playwright for XSS) to exercise each vulnerability.
  \item Captures detailed logs of payload transmission, script execution, and file retrieval.
  \item Surfaces gaps in Scrapy’s default defenses and suggests mitigation layers.
\end{itemize}
